\input book

\centerline{\bf Distance}
\centerline{\it Sunday, 2017/03/19}



\vfill
\break



The traditions of philosophy (and religion and education, inclusively)
are far more than opposition to the pursuit of satiation and
resistence to growth.  

The traditions of philosophy are far more than the tools to identify
and characterize elements of individual and social experience.
Through the power inherent in the traditions of thought and
consciousness we have traversed the oceans of outer space to walk on
the Moon.

The dependence of the individual on the experience of expectation
binds us to the overt and subvert stories and projections that help
and harm.  When we apply the capacity for learning and growth to
experience, we elevate ourselves into those powers of consciousness
that have taken us to the Moon.  We may focus on issues of being or
technology as critical to existence.  

And yet we require the constant reminders of distinction and
differentiation, substance and superficiality.  One may reasonably
wonder how this state of affairs is possible.  Indeed, I am often
confronted with the implausible.  In effect, I am asked to indict
absurdity, again, on a daily basis.

In exchange I present the world of art in contemporary communications
media.  Music videos, cinema, commercial advertising all speak the
language of art and commerce of contemporary communications media.  

In na\"{\i}vet\'{e}, a commercial presentation is no more than
``industrial video''.  It fails to raise itself into a conscious mode
of expression.  This mode of presentation may be appropriate to the
communication of a simple fact, like the image of a physical object.
The reader has no other use of the presentation than to read the
(visual) identity of a physical object.

Otherwise it is tedious, lacking recognition of basic facts of natural
language.  In this case, mortal errors of judgement are committed
against the capacity of the {\it homo sapien} for recognition,
abstraction, and imagination.  Such errors could start a war, for
example.

In art, the construction and use of expressive context preceeds the
assertion or interrogation of the abstractions formed by expression.
The imagination of the reader is engaged, regardless of the
willingness of the reader to admit the engagement.  

The producers of the communication, film makers, have an intent that
may directly oppose such a na\"{\i}ve interpretation as held in the
denial of art.  The classic example is a presentation about war
intended to raise consciousness about history and experience, but
taken to be a stimulant in favor of aggression.

Even in this typical phrasing we employ the phrase ``taken for'' in
misunderstanding rather than the pretence of innocence in
misunderstanding which one would represent as ``understood as''.  

With the simple interruption of the projection of absurdity that
provides stimulation we awake to the natural awareness of self, body
and mind, in which war is obviously not healthy.

\vskip 0.25in

The history of visual language in cinema, in motion pictures, is as
old as the medium, itself, repleat with positive and negative cases.
Families of expression and language have emerged with genre of subject
and study to fulfill the needs of expression over kinds and degrees of
consciousness.

The medium regularly escapes its own containment, requiring the reader
to consider film makers and novelists in their views and commitments.
In my own case, the film ``Eyes Wide Shut'' required a trip to the
bookstore to dig up the work of the novelist in order to restore my
confidence in my world view.  

Otherwise it's all relatively self explanatory under the obligation of
the reader to consciousness.  If the producer intended the wrong idea,
then the producer suffers the fate of irrelevance.  If the producer
intended the right idea with respect to the history of consciousness
and the value of life, then the producer deserves at least the reward
of relevance.

\vfill
\bye
