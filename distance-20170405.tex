\input book

\centerline{\bf Distance}
\centerline{\it Wednesday, 2017/04/05}



\vfill
\break

The world is physical, or metaphysical.  The subjective being, person,
may be trained to either.  The untrained person is subjective, living
immersed in a metaphysical universe.  The technology delivered by our
human ambition populates this universe with the alloys of lead, gold,
copper, tin, mercury, silver, aluminum, lithium, every element known
and many unknown. \break Some have power.  Some, voice.  All, breath.

Each person, another world.  Each expression, another language.  Our
babies cry in confusion.  Some adults cry in confusion.  Sometimes a
crying out is necessary.  Most times, it is a question unknown that
none can answer.

When it is healthy, most are content and happy.  When it is unhealthy,
there may be war.  The habit of war is a slavery, as is the habit of
the tolerance of a poverty of education.

\vfill
\break

Ir\`{e}ne's windows lit the sky.  The light that filled my eyes was
that sense of contrast that I derived from her sense of sight.  Her
transparency, clairvoyance.

A rainy day, cherries or plums just popping.  Clouds full of billowed
textures of grey.  Strings of grapes laid on the ground as lifeless
branches.  The windows I sat before looming over head, monuments to
chairs.  Large rectangles of glass panes that could, in theory, easily
pass the largest motorcycle or the smallest car.

Obviously the impact of Ir\`{e}ne's pages derived from the
foreknowledge of the writer's personal and historical context.  That
she could write any such thing, a miracle.  While others were chewing
fingernails, she composed a refuge with pen and ink that formed a
shelter of justice.  Her ink in the state of those fingernails, torn
from her, and in that indictment, justice.

The crows and jackals of her time so similar to those of my own.
Unlike Ir\`{e}ne, I have cut their flesh into bite sized chunks and
fed them to the fishes.  The illness of the hyper-rational denial of
metaphysical fact.

Obviously this moment is far more subtle than that of Europe during
World War Two, or the South of \break France in 1941.  The psychological
structures are the same, but the American fascist malaise of the
insecure who band together to squelch the sense of their illness has
not manifested as badly.  The hyper-rational denial of self combines,
here, with the materialization of identity in contrast to the European
actualization of identity.  In our case, the self error is more
contrasted.  The ill know, and the error manifests outwardly through a
society that has not entirely faced the immersive qualities of
intellect.

The crutches are words.  The guilty parties include, most notoriously,
``reality'', ``money'', and \break ``competition''.  The collective
unconscious is stressed by the experience of novel population and
technology effects.  In all fairness, it is a proper case of
intellectual immersion -- in the sense that self awareness on the
social and cultural scale have not yet confided in a perspective or
independence of intellect, independence of expression from conception,
and the separation of faith from na\"{\i}vet\'{e}.

In other words, we have forgotten how to speak in our generations that
have experienced a world dominated by its sensitivity to time.
Ironically, it is a mature perceptual frame that is partly responsible
for the dissolution of the juvenile and immature frames of
understanding.  An immature or insecure sense of maturity that
confides in the skill to manage time with the intimacy of a second,
while subject to an increasing sense of expiration.

It is precisely these metaphysics that we need to institutionalize, to
culture and develop, teach, discuss, open, maintain, and preserve.  We
know the value of the independence of thought and objective, the
fairness that preserves opportunity or occasion.  We need to develop a
far better sense of our personal metaphysics, and the metaphysical
world they produce.

First, interpersonal interaction.  We indulge love and affection and
combat for audiences addicted to an unknown immersion, rather than
develop a conscious reflection on the immersions we are producing, and
-- personally and collectively -- have produced.

Second, information handling.  We indulge interpretation for audiences
in need of condensations and reflections with far too small a sense of
propriety, of independence of subjective and objective and self and
other.  The method of alternatives has demonstrated the value of
fairness of occasion for millennia, while the application of the
derived discipline remains a quality of scarce appearance.  That is,
scientific discipline and its derivatives including journalistic and
judicial integrity.

Until we recognize the metaphysical skills necessary to a healthy
metaphysical world, we will not have a healthy metaphysical world.
Likewise, until we recognize the responsibility for a healthy
metaphysical world, we will not have a healthy metaphysical world.
And, following, until we recognize that the health of the metaphysical
world is proportional to this metaphysical \break wealth, we will
continue in the vanity of an intractable metaphysics.

Briefly, most concisely, we are afraid of our own evolution.  Like any
immature adult, we cling to na\"{\i}vet\'{e} waiting on the evidence
of the experience of others.  The point of conception, realization and
actualization is individual.  It requires faith in the independence of
the mind-body identity from material objectivity, knowledge of the
human condition as first subjective, and knowledge of the distinction
of subjective and objective as emotional and intellectual, or
spiritual and rational.  With these facts we understand the necessity
of self balance to the pursuit of opportunity or occasion.  And,
further to the point, the importance of self awareness.

\vfill
\bye
