\input book

\centerline{\bf Distance}
\centerline{\it Tuesday, 2017/03/28}



\vfill
\break

``Why should a child need to have a father?''

I don't want to say that on the air.  It's too pointy.  There's no
space for the discussion of anything in this world.  It's all hit and
run.

``In the perpetual sense, I think you mean.''

``Yes, of course.  Obviously in the biological sense there's the
mechanics of contribution.  But in the cultural space there's this
Reality that demands that a child have a father.''

``Otherwise the child is taken.''

``Yes, to be plain about it.  In one sense or another.''

``The subjective probability of success in the collective
consciousness falls.''

The exploitation of mass media had contributed to the massification of
the individual.  The individual as slave to society and its media.  It
had done so by an error of youth.  Decades of broadcasting driven by
materialized identities.  Broadcasting the socio-economic realization
of a socio-genetic actuality.

\vfill
\break

``The expectation of freedom is traded, in some quarters, for the
absence of expectation.  For the freedom to be null and void, inert,
inutile, meaningless, and pointless.''

``The positive expectation is traded for the negative expectation.''

``I'm not sure the negative expectation has a sign.  I think it's
vacuous.''

``It is darkness and poverty and the absence of \break merit, and in this
constructed sense it is inert and meaningless.''

``Yes.  Steven King's Langoliers.  The dimension that exists between
the cracks.  The poverty of education that exists in this world.''



\vfill
\break

``We have the habit of not discussing it.''

``We have such a habit of avoiding our own discomfort that we've
become slaves to it.  Slaves to comfort.  Slaves to a world of comfort
that's so unhealthy as to be most acccurately described as the
perversion of comfort, the perversion of the status quo.''

``In the sense that we've become slaves to society in place of the
recognition that the health and well being of society depends on the
health and well being of the individual.''

``Yes, and that the health and well being of the individual with
respect to these social issues is primarily a matter of education.''

``Education is the gap between the haves and the have-nots.''

``Yes.  Of course.  The facts and understandings that differentiate
the world of yesterday from the world of today.''

``And the world of tomorrow.''

``Well, let's catch up with the twentieth century before we go off
exploring.''




\vfill
\break


The camera went dead and an echo spoke the words that said the studio
was vacated.  She smiled faintly and put a hand on his.  They were
both a bit tired, and a bit sick of listening to themselves talk and
speak.  He looked up from the table where he had been collecting some
things, and himself.  His distance spoke fondly of their
accomplishment.

She stood, turned, and walked out of the funny room with desk and
camera.  His face turned to a grin of thanks.  They were reclaiming
the sense of time and space, the threads of life beyond their
respective careers.


\vfill
\break


She walked up the avenue toward home, a prayer for consciousness, a
prayer for the salvation of life from the desolation of waste.  That
she had done the right thing with her time.  That she had not wasted
her breath on this rather some alternative unknown to her.

She looked at the sky.  Clouds dimly lit in the early evening.  She
wondered at their glory.  The sidewalk under her feet begged
forgiveness for covering the earth.


\vfill
\break


As she unpacked her groceries she thought of the rain.  It had begun
as she stepped off the sidewalk.  Enough to spell the garden another
week.  

A moment later, when it had stopped, she opened the door to the garden
as rays of sunset lit her flowers with their life.


\vfill
\bye
