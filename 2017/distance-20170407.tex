\input book

\centerline{\bf Distance}
\centerline{\it Friday, 2017/04/07}



\vfill
\break

It could be said, in a small sense, that my experience of loving women
taught me little more than the fact -- in my case -- that sex doesn't
lead to relationships.  And, it could be said, with the same loss of
objective determinism, that without sex I would not have learned this
fact.

I imagine this is true because I had a need to learn this, or perhaps
there is a web of reasons why my experience of learning about my heart
included sex.  That I believed that relationships and learning evolved
from experience, and this experience includes sex as part of the
subject matter.  It is objective, this belief, and the world of my
awareness has had more than a healthy or even average degree of
objectivity.  If one is to exit an immersion, one must get a handle on
its dimensions.

And while I've been relatively conservative, I've had enough nerve to
learn by first hand experience what I could not establish otherwise.

My na\"{\i}ve sense of the world was neither completely lost nor
completely found.  It was a mess, reflecting the state of a world
largely unaware of itself.

Perceptions of the critical and the indulgent varied by fractions that
first appear individual to the na\"{\i}vet\'{e} of inexperience.  Or
at least this was a seemingly valid frame of reference that held some
interest for me.

In this world, it is far more useful to see people as no more than
individuals, to not attach more than observation, to not attempt to
acquire the benefit of others' experience by observation.

The benefit of experience by observation is far better traded for the
benefit of a clear objective to weave one's personal, subjective
exploration into.

In the case of sex and love and relationships, the objective is not
reproduction -- which is possessed of the frailty of the immediacy and
immersion of the body -- but the dimensions and magnitudes of one
self.  Heart, ambition, and courage are well known concepts that
translate directly into relationships of all shapes and sizes.

In these subjective dimensions, the mate is very close.

\vfill
\break

In the absence of a firm socio-cultural awareness of metaphysical
fact, we find ourselves bound to Freudian problems.  The burden of the
parental r\^{o}le on the child is not overcome.  Indeed, we forcefully
instill an association of parental control within the subjective
theater of socio-economics, generally.  The so-called ``fear of god''
is a placeholder in desperate need of replacement.  The misassociation
of focus and discipline with control and tedium and pessimism is born
out of these issues.

Proportional examples include an undue mistrust of alternative sources
of learning and information, misplaced senses of recognition of rights
or permissions, and generally misplaced senses and associations within
the space of metaphysical fact as results in a relatively burdened or
brutal emergence of the individual from immaturity.

Our present sense of population has pressed these processes into the
awarenesses that demand our attention.  We perceive a need for greater
awareness and peace.

The socio-cultural dominance of objectivity over subjectivity is
evident in the surplus of a sense of economics and a deficit of a
sense of well being and art, and, to face the extremity of the
problem, science.

The subjective universe that objective fact is component to demands
balance.  That the tools of objectivity including rationality have
become imbalanced to their benefit is clear from the absence of
science in a world stuck to the windows of objectivity.

Finding our balance requires the freedom of \break informed individual
evolution.  The continuous development and exposition of metaphysical
fact is as necessary, rather more necessary to this life than the
development and presentation of physical fact ({\it i.e.} new
physics).

Obviously, neither physical nor metaphysical development need suffer,
but finding our will to develop the metaphysical and subjective as
important as the physical and objective is monumentally important.

\vfill
\break

The individual is primarily subjective, and applies training, skill,
and discipline to the realization of objective understanding and
comprehension.  The primary example of this process is in the
acquisition of mathematics.  Fluency with the languages and tools of
mathematics leads to mathematical maturity as the capacity for
mathematical thought, understanding, and comprehension.

Similarly, the acquisition of musical training and skill and
discipline is a process of developing a capacity for reading and
performing and writing music.  It is an entirely distinct mode of
thought and expression from the human natural languages of verbal
speech.

The history of the sciences shows many rivers of thought flowing into
an ocean of mathematics and \break physics, logic, and the application
of these disciplines -- tools -- to human subjectivity.  The
independence of these tools from the inert confusion or ignorance that
preceded them illustrates the value of these tools.

Today, the metaphysical independence of the individual is relatively
unknown.  Our physical independence is circumscribed by our economics,
an observation which places us far removed from alternatives to
technological sustenance and development.  A metaphysical independence
would permit our children knowledge and understanding as would obviate
or void ``the meaning of life'', dispel ``questions regarding sex'',
and eliminate ``the chaos of politics, law, and war''.

For millennia we have developed the tools for this independence, as
recorded to the history of thought.  We have developed languages to
facilitate the fairness of occasion in both the physical and
metaphysical domains, and are found needing little more than a
consolidation of these efforts.

\vfill
\bye
