\input preamble

Irène is clairvoyant.  I am a brick head who rejects the sight of most things.  She has a woman’s talent for the acceptance of people, for love unencumbered by the mysteries society piles onto its males.  A woman is left with her natural wealth, while a male is beat upon and out cast as if criminal, guilty of nothing until innocent of everything.  Conversely, it is the male and the intelligence of genetic success that tortures the female, robbing her of that wealth.  His true nature is rather asymmetric and slight.  He would hold her in preservation, share her burdens and glories as partner.  As a half of one.  Like two halves of an avocado, one holds the pit and the other the space the pit occupies -- under the analysis of the avocado.


What am I afraid of, losing her?  She is lost, according to that record and my own intellect.  She is won according to my soul, the one she wrote into the palm of my hand.  Her letters betray me.  It cannot stand.  She is on her back.  I am not.  Her plays too -- dark, disingenuous, dangerous, worldly, other than anything I would want to know as intimately as the steps I want can allow.  Therefore, they are not my steps to take.  They are hers.  It is the actor who stands before the audience holding the writer’s pen, wearing the art.  The sights beheld in this invisible theater of the implication of the dark reflections on the visible much worse to my mind than Irène’s casual billet making with her temporal destruction at the hands of the Nazis.


She writes, in that mist.


As you know, dearest husband, I’ve raised up your world of heaven and hell over my head that they might turn to stone with my step to you.


The terror of my existence, the world she has immersed me into, is how long will it take to satisfy her requirements.


The children call out, into the air that I can hear, “your ass is so fat”.  I can only imagine an allusion to the wealth that fills their eyes.  The world of communications saturated with reflections direct and indirect, light and dark, the successes and failures of millions of lives -- billions of lives -- played out by greed of seconds.


To my mind it never ends, never concludes.  She is lost to it, to art, to the world.  To people who gasp at air like fish drowning out of water, entirely unaware of themselves and even less concerned with our fate in this play.


I am to understand quite the opposite, however.  That it is my own allusions, my own terror at the fickle whimsy of our present economic nature, that is mistaken.  That sight does not require more than the lifting of eyelids, easily done.  Which is what I meant by gasping at greed.  The world is won.


Then why am I here.  Because she is no matter to it, and every matter to it.  And I am neither here, nor not here, alive not dead.


How do you show a person their eyes filled with the world that exists as a series of successes and failures, millions per second?  There is no requirement beyond that compassion and dignity we might call a sense of humanity.  Whether a success or failure fills the air we breathe is not important.  The implication for the next minute, hour, day, and year, however, is.


So few, it seems, have the ability to lift a head up to such a height.
\bye
