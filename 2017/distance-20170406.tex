\input book

\centerline{\bf Ir\`{e}ne}


\vfill
\break

It was empty, her meaning, and in that the grip of death.  Her pages
blank, her ink dry.  Absent a spark of appreciation.

Her name is Ir\`{e}ne.  She's been dead longer than my mother has been
alive.  But tonight she is my lover.  Somehow, in the reaches of
experience, her absence is mine.  It is not mine, it is like mine, and
in this interplay is the subtlety of love making.

Unlike her, I populate her blank pages with the wildlife that escapes
the enemy of humanity.  I don't need to call that a deity.  It is all
around me, in me, the breath of life.  But like her, I honor it with
an element of smoke that fouls my body and the air but occupies the
crack in space that leaves me separate from the life that fills my
heart.

\vfill
\break

Another number advances my page, my fingers, my lips and my arms.  My
eyes water under her.  My arms and hands feel light with her weight.
Ir\`{e}ne.  I feel her blank pages and lost books, friends unknown to a
heart at sea under the pitch of it, sitting in the comforts of my
sheltered life.  A cigarette rolled with tabacco found on the side
walks of this modest, heart felt town carved in stone out of mountains
lost in time.  To me they are lost in time because I am lost to time.
My age advances with the soil and toil under a celestial heaven known
within but to whom other?  A question mark known to me, but to whom
other.  A paragraph kept in place for time held too dearly that
breath, each, might be appreciated.

With all the daring available I pick up her book, turn its pages in
review.  Losing nothing that I might have read, if possible.  Holding
her as a knave holds a word.  Too closely.  Too tightly.  My grip
unable to loosen, locked.  I am scared.  Terrified.  Terrified to be
lost to loss.  To senselessness without intent of purpose, commitment
of love.

Sleep is a sound cure.  In sleep I find peace where my body escapes
time.  I levitate above the din of it, in love and spirit held dear.
My font within.  The source of pleasure and this morning's inevitable
pen.

\vfill
\break

Somewhere within the masculine, under the fear, apart and separate
from the need to defend the self, is the divorce of the sex writ on
the body as the freedom from the sex.  The body.  The penile thing
between the legs that gives cause, that is disposed of as cause of the
pain of time and place and role and purpose and intent and meaning.
From here, this spiritual abstraction that divorces the life and the
world with it, he wonders if she does the same.  The feminine.  Woman.
Any woman he might be able to imagine, perhaps to meet.  To speak to.
To divorce and free and to live with.  To find peace in the world.  As
he finds peace under and outside the world, in that bed where he
sleeps.  Without sex.  With love that has no sex.  The love from
within where dreams have no care.

A bird sings in the morning of this spilled ink.  I am under the bird
in my incomplete wakefulness with Ir\`{e}ne beside me, in pages in a
book.

I can feel her, like I can feel the cursor slipping and spotting the
letters under my fingers on the miracle of technology in my hands.  My
mechanical pen.  The one that connects to the waking world of sight
and insight where context would place me as a book on a shelf in a
library.

Ir\`{e}ne remembers her experiences from the unknowable distance to
the place she has become.  Desolate, losing, not yet lost.  Finding,
as one breathes, another word.

\vfill
\break

On her sixteenth page she picks us up, both of us.  I have little
right to lie in her bed with her, the way I do.  She is far more
beautiful than I, and I am far too sensitive to her value to me to
ignore any quality of hers.  She is dead, and I am alive.  To begin
with.  But there, on that page, my love, have you spelled us both.
And a great many others, all the same, ideally.  Those in places like
these, especially.

The keyboard clacks under my fingers, it can pick but not swipe.
Picking and putting works half as well under my arms than a real
keyboard, but swiping its cursor to move, remove and replace a mispent
or absent glyph more a tragedy than the comedy that is my skill with
typing.

My love, stop crying.  It is I.  You know this place, this heart human
too well.  It is true.  You have found me out.  I think of death, I
wonder if I might have the purpose to end this existence before they
do.  To deny the corpse that passage.

As you know, already, I have not made that choice.  And you, my love,
understand why.

The air in here, so well sealed from the elements, grows stale with
the hour.  The sacrificial incense colors rays that wake it.

I have misplaced a digit where once I would not have let half a second
see it.  This is my third page, this morning, not my fourth.

I've cracked open a door as the clear sky and air not cold enough
threaten my habit.  It will become too hot to stay here, with no
shade.  They have taken the shade.

I've been very selfish.  That is what I saw last night.  The sea had
hollowed out the shells of an expanse of great mansions.  The remains
of an exploit.  A beautiful hike I was telling you about.  It led to a
beach through a grove of tall, thin, sky topped pines.  First a great
old stage remained apparent on a crest of land that rose above the
sand, abandoned by the tide.  Then another beach, and a great old
house of concrete block I should not wonder.  Ornate enough.  Even
handsome, once.  Alone, a sight.  Sitting on sand, intact save the
habits of barnacles and their friends.  The horizons visible through
the wreck, dramatic.  Another beach, another house.  And then a string,
a turn, laid out as if on an avenue and a pier.  Standing where the
sea would return, again and again.

I opened another door with the second piece of wood.  That I might
save some time, captured in the mind as a moment does.  Coherent in
spirit, known within and without.  Sensible to touch more than sight,
ear more than mind.

This strange, temporary existence plays -- preys -- on my heart.  My
eyes open to the sights around me.  In front of me a high way.  And
behind me a few small houses planted nicely along a structure bedded
with lawnery.  An image of comfort contrasts to an image hostile to my
body, and hostile to my mind, my knowing and being and seeming.  How
could it be that I am here, like this.  Tenuously established
temporaneously.  The heart quickens with the incertanty of it.  The
ultimacy of this, among the colors of the air.  To be, or not.  And if
not, then not.  I am aware of no fate.  And this causes me no small
stress in my little chest.  My sense of my own future lost to it.

It is not ground.  It is no place in particular.  And yet, there is an
existence, of a kind, that is mine to have and to hold.  Like this,
better than that held by Ir\`{e}ne.  Its similarities held remotely,
somewhere else.  Known remotely.  Assumed remotely.  Ideally.  Perhaps
uncertanly.  There remains, today, some whisp of a binding to that
former uncertanty.

And, unlike Ir\`{e}ne, I can sleep.  I can pull a cover over my little sky
light and pull a cover over my arms, close my eyes and raise my head.

\vfill
\break

Ir\`{e}ne is clairvoyant.  I am a brick head who rejects the sight of most
things.  She has a woman's talent for the acceptance of people, for
love unencumbered by the mysteries society piles onto its males.  A
woman is left with her natural wealth, while a male is beat upon and
out cast as if criminal, guilty of nothing until innocent of
everything.  Conversely, it is the male and the intelligence of
genetic success that tortures the female, robbing her of that wealth.
His true nature is rather asymmetric and slight.  He would hold her in
preservation, share her burdens and glories as partner.  As a half of
one.  Like two halves of an avocado, one holds the pit and the other
the space the pit occupies -- under the analysis of the avocado.

What am I afraid of, losing her?  She is lost, according to that
record and my own intellect.  She is won according to my soul, the one
she wrote into the palm of my hand.  Her letters betray me.  It cannot
stand.  She is on her back.  I am not.  Her plays too -- dark,
disingenuous, dangerous, worldly, other than anything I would want to
know as intimately as the steps I want can allow.  Therefore, they are
not my steps to take.  They are hers.  It is the actor who stands
before the audience holding the writer's pen, wearing the art.  The
sights beheld in this invisible theater of the implication of the
reflections on the visible much worse to my mind than Ir\`{e}ne's
casual billet making with her temporal destruction at the hands of the
Nazis.

She writes, in that mist.

As you know, dearest husband, I've raised up your world of heaven and
hell over my head that they might turn to stone with my step to you.

The terror of my existence, the world she has immersed me into, is how
long will it take to satisfy her requirements.

The children call out, into the air that I can hear, ``your ass is so
fat''.  I can only imagine an allusion to the wealth that fills their
eyes.  The world of communications saturated with reflections direct
and indirect, light and dark, the successes and failures of millions
of lives -- billions of lives -- played out by greed of seconds.

To my mind it never ends, never concludes.  She is lost to it, to art,
to the world.  To people who gasp at air like fish drowning out of
water, entirely unaware of themselves and even less concerned with our
fate in this play.

I am to understand quite the opposite, however.  That it is my own
allusions, my own terror at the fickle whimsy of our present economic
nature, that is mistaken.  That sight does not require more than the
lifting of eyelids, easily done.  Which is what I meant by gasping at
greed.  The world is won.

Then why am I here.  Because she is no matter to it, and every matter
to it.  And I am neither here, nor not here, alive not dead.

How do you show a person their eyes filled with the world that exists
as a series of successes and failures, millions per second?  There is
no requirement beyond that compassion and dignity we might call a
sense of humanity.  Whether a success or failure fills the air we
breathe is not important.  The implication for the next minute, hour,
day, and year, however, is.

So few, it seems, have the ability to lift a head up to such a height.

\vfill
\break

In this existence, in this world, money is virtually as immersive or
ubiquitous as air.  One's dependence on money is virtually as certain
as one's dependence on air.  In order to see this, one must be able to
step away from the psycho-emotional planes of existence and take the
issue in hand, into a frame of thought or analysis that is independent
from the mind-body identity.  Otherwise one is slave to it.

It takes multiple decades of life time to achieve the complexity of
mind necessary for this subtlety.  Otherwise, one is immature,
juvenile, a babe in the woods, immersed in a reality which one only
suspects but does not grasp.  Has no handles on.  One is naked under
the ocean of the subject and its critical issues.

What is moral good?  What is moral wrong?  In the case of money, and
its links into politics and government, many fail these tests daily.

In these moments of history characterized by systems of self
government, it is these issues that have delivered the great monuments
of history.


\vfill
\break

The world is physical, or metaphysical.  The subjective being, person,
may be trained to either.  The untrained person is subjective, living
immersed in a metaphysical universe.  The technology delivered by our
human ambition populates this universe with the alloys of lead, gold,
copper, tin, mercury, silver, aluminum, lithium, every element known
and many unknown. \break Some have power.  Some, voice.  All, breath.

Each person, another world.  Each expression, another language.  Our
babies cry in confusion.  Some adults cry in confusion.  Sometimes a
crying out is necessary.  Most times, it is a question unknown that
none can answer.

When it is healthy, most are content and happy.  When it is unhealthy,
there may be war.  The habit of war is a slavery, as is the habit of
the tolerance of a poverty of education.

\vfill
\break

Ir\`{e}ne's windows lit the sky.  The light that filled my eyes was
that sense of contrast that I derived from her sense of sight.  Her
transparency, clairvoyance.

A rainy day, cherries or plums just popping.  Clouds full of billowed
textures of grey.  Strings of grapes laid on the ground as lifeless
branches.  The windows I sat before looming over head, monuments to
chairs.  Large rectangles of glass panes that could, in theory, easily
pass the largest motorcycle or the smallest car.

Obviously the impact of Ir\`{e}ne's pages derived from the
foreknowledge of the writer's personal and historical context.  That
she could write any such thing, a miracle.  While others were chewing
fingernails, she composed a refuge with pen and ink that formed a
shelter of justice.  Her ink in the state of those fingernails, torn
from her, and in that indictment, justice.

The crows and jackals of her time so similar to those of my own.
Unlike Ir\`{e}ne, I have cut their flesh into bite sized chunks and
fed them to the fishes.  The illness of the hyper-rational denial of
metaphysical fact.

Obviously this moment is far more subtle than that of Europe during
World War Two, or the South of \break France in 1941.  The psychological
structures are the same, but the American fascist malaise of the
insecure who band together to squelch the sense of their illness has
not manifested as badly.  The hyper-rational denial of self combines,
here, with the materialization of identity in contrast to the European
actualization of identity.  In our case, the self error is more
contrasted.  The ill know, and the error manifests outwardly through a
society that has not entirely faced the immersive qualities of
intellect.

The crutches are words.  The guilty parties include, most notoriously,
``reality'', ``money'', and \break ``competition''.  The collective
unconscious is stressed by the experience of novel population and
technology effects.  In all fairness, it is a proper case of
intellectual immersion -- in the sense that self awareness on the
social and cultural scale have not yet confided in a perspective or
independence of intellect, independence of expression from conception,
and the separation of faith from na\"{\i}vet\'{e}.

In other words, we have forgotten how to speak in our generations that
have experienced a world dominated by its sensitivity to time.
Ironically, it is a mature perceptual frame that is partly responsible
for the dissolution of the juvenile and immature frames of
understanding.  An immature or insecure sense of maturity that
confides in the skill to manage time with the intimacy of a second,
while subject to an increasing sense of expiration.

It is precisely these metaphysics that we need to institutionalize, to
culture and develop, teach, discuss, open, maintain, and preserve.  We
know the value of the independence of thought and objective, the
fairness that preserves opportunity or occasion.  We need to develop a
far better sense of our personal metaphysics, and the metaphysical
world they produce.

First, interpersonal interaction.  We indulge love and affection and
combat for audiences addicted to an unknown immersion, rather than
develop a conscious reflection on the immersions we are producing, and
-- personally and collectively -- have produced.

Second, information handling.  We indulge interpretation for audiences
in need of condensations and reflections with far too small a sense of
propriety, of independence of subjective and objective and self and
other.  The method of alternatives has demonstrated the value of
fairness of occasion for millennia, while the application of the
derived discipline remains a quality of scarce appearance.  That is,
scientific discipline and its derivatives including journalistic and
judicial integrity.

Until we recognize the metaphysical skills necessary to a healthy
metaphysical world, we will not have a healthy metaphysical world.
Likewise, until we recognize the responsibility for a healthy
metaphysical world, we will not have a healthy metaphysical world.
And, following, until we recognize that the health of the metaphysical
world is proportional to this metaphysical \break wealth, we will
continue in the vanity of an intractable metaphysics.

Briefly, most concisely, we are afraid of our own evolution.  Like any
immature adult, we cling to na\"{\i}vet\'{e} waiting on the evidence
of the experience of others.  The point of conception, realization and
actualization is individual.  It requires faith in the independence of
the mind-body identity from material objectivity, knowledge of the
human condition as first subjective, and knowledge of the distinction
of subjective and objective as emotional and intellectual, or
spiritual and rational.  With these facts we understand the necessity
of self balance to the pursuit of opportunity or occasion.  And,
further to the point, the importance of self awareness.

\vfill
\bye
